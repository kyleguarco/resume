\documentclass{resume}
\usepackage[left=0.4in,top=0.4in,right=0.4in,bottom=0.4in]{geometry}

\newcommand{\tab}[1]{\hspace{.2667\textwidth}\rlap{#1}}
\newcommand{\itab}[1]{\hspace{0em}\rlap{#1}}
\name{Kyle Guarco}

\address{
	\href{tel:+18605059368}{(860) 505-9368} \\
	Plainville, CT
}
\address{
	\href{mailto:kyleguarcocs@gmail.com}{kyleguarcocs@gmail.com} \\
	\href{https://www.linkedin.com/in/kyleguarco/}{linkedin.com/in/kyleguarco} \\
	\href{https://github.com/kyleguarco}{github.com/kyleguarco}
}

\begin{document}

%----------------------------------------------------------------------------------------
%	EDUCATION SECTION
%----------------------------------------------------------------------------------------
\begin{rSection}{Education}
	{\bf Bachelor of Computer Science}, Central Connecticut State University \hfill {Sep 2019 \-- May 2023}\\
	{\bf High School Diploma}, Plainville High School \hfill {Sep 2015 \-- June 2019}
\end{rSection}

%----------------------------------------------------------------------------------------
% TECHINICAL STRENGTHS
%----------------------------------------------------------------------------------------
\begin{rSection}{SKILLS}
	\begin{tabular}{ @{} >{\bfseries}l @{\hspace{6ex}} l }
		Languages & C, JavaScript, Python, RustLang, Java, Lua\\
		Frameworks & React Native\\
		Technical Skills & Linux, Podman Containerization, PostgreSQL, LibreOffice,\\
		& Wordpress CMS, Git, GNU Privacy Guard, Fiery, PapercutMF\\
	\end{tabular}\\
\end{rSection}

\begin{rSection}{EXPERIENCE}
	\textbf{Student Production Technician} \hfill {Sep 2020 - May 2023}\\ % chktex 8
	Central Connecticut State University
	\begin{itemize}
		\setlength{\itemsep}{-3pt}
		\item Ran lighting and sound for shows in Welte Auditorium at CCSU
		\item Provided technical support for events which involved live-streaming and presentations
		\item Led a team of 2-3 people to accomplish logistical tasks
		\item Optimized a bi-annual process from 3 weeks to 6 hours
	\end{itemize}
	
	\textbf{Copy Center Manager} \hfill {Aug 2022 - Nov 2022}\\ % chktex 8
	Central Connecticut State University
	\begin{itemize}
		\setlength{\itemsep}{-3pt}
		\item Trained new staff on customer service, Fiery and PapercutMF
		\item Maintained a grayscale printer, a color printer, and 3 computers
		\item Made monthly income reports
		\item Made service calls and ordered paper
	 \end{itemize}
\end{rSection}

%----------------------------------------------------------------------------------------
%	WORK EXPERIENCE SECTION
%----------------------------------------------------------------------------------------
\begin{rSection}{PROJECTS}
	\vspace{-1.25em}

	\item \textbf{Kryptos 2022 - Computer Security} \hfill {Python}\\
	\vspace{-1.25em}
	\begin{itemize}
		\setlength{\itemsep}{-3pt}
		\item Worked to solve cryptography challenges for Central Washington University's Kryptos 2022
		\item Part of my final project on studying different cryptographic algorithms
		\item My partner and I obtained one of the top spots for solving all three challenges
	\end{itemize}

	\item \textbf{Snake Game - Systems Programming} \hfill {C, ncurses, Git}\\
	\vspace{-1.25em}
	\begin{itemize}
		\setlength{\itemsep}{-3pt}
		\item A snake game written in C using the ncurses TUI library
	\end{itemize}

	\item \textbf{FootLocker Web Scraper - Secure Software Systems} \hfill {Java, jsoup, Git}\\
	\vspace{-1.25em}
	\begin{itemize}
		\setlength{\itemsep}{-3pt}
		\item A web scraper that grabs information on new shoes from FootLocker
		\item Parses the sitemap to navigate shoe categories
	\end{itemize}

%	\item \textbf{Lua Parser - Personal Project} \hfill {Rust, Lua, nom, Git}\\
%	\vspace{-1.25em}
%	\begin{itemize}
%		\setlength{\itemsep}{-3pt}
%		\item Parses a Lua 5.4 file into an abstract syntax tree
%		\item My next goal is to create a compiler for Lua using this parser
%	\end{itemize}
\end{rSection}

%----------------------------------------------------------------------------------------
%\begin{rSection}{Extra-Curricular Activities}
%	\begin{itemize}
%	\end{itemize}
%\end{rSection}

\end{document}

